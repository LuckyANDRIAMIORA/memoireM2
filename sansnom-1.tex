\documentclass[12pt]{report}
\usepackage{graphicx}
\usepackage{array}
\usepackage{tabularx}
\usepackage{amsmath}
\usepackage{hyperref}
\usepackage{geometry}
\usepackage{fontspec} % For custom fonts
\usepackage{xcolor}
\usepackage{float}
\usepackage{titling}
\usepackage{tikz}
\usepackage{fancyhdr} % For custom headers/footers
\usepackage{tocbibind} % To include TOC, LOF, LOT in TOC
\setmainfont{Times New Roman} % Set the main font to Times New Roman
\geometry{a4paper, margin=1in}
\usepackage{setspace}
\onehalfspacing 
\usepackage{titlesec}
\usepackage{fancyhdr}
\pagestyle{fancy}
\fancyhf{} % Clear all header and footer fields
\fancyfoot[C]{\thepage} % Place the page number in the center of the footer
\renewcommand{\headrulewidth}{0pt}  % Removes the line in the header
\renewcommand{\footrulewidth}{0pt}  % Removes the line in the footer
\hypersetup{
   colorlinks=true,
  linkcolor=black,   % Set link color to black
  filecolor=magenta, 
  urlcolor=cyan,
  pdfborder={0 0 0} 
}
\renewcommand{\contentsname}{SOMMAIRE GENERAL} 
\renewcommand{\listtablename}{LISTE DES TABLEAUX}
\renewcommand{\listfigurename}{LISTE DES FIGURES}

\titleformat{\chapter}[display]
    {\normalfont\large\bfseries\centering}
    {\chaptertitlename\ \thechapter}
   {20pt}
    {\Large}

\titleformat{\section}
    {\normalfont\bfseries} 
    {\thesection}
    {1em}
    {}

\begin{document}	
			 \pretitle{
				\begin{tikzpicture}[remember picture, overlay]
			   		\node[anchor=north west, xshift=1.5cm, yshift=-1cm] at (current page.north west) {
        						\includegraphics[width=2.5cm]{image1.png}
					};
    					\node[anchor=north west, xshift=4cm, yshift=-1cm] at (current page.north west) {
        						\includegraphics[width=2.3cm]{image2.png}
					};
					\node[anchor=north east, xshift=-1.5cm, yshift=-1cm] at (current page.north east) {
        						\includegraphics[width=5cm]{image3.png}
					};
				\end{tikzpicture}
				\begin{center}
					\textbf{\large UNIVERSITE DE FIANARANTSOA ECOLE NATIONALE D'INFORMATIQUE \\[0.5cm] MEMOIRE DE FIN D'ETUDES POUR L'OBTENTION DU DIPLOME DE MASTER PROFESSIONNELLE}
					\\[0.5cm]
							\textbf{\underline{Mention:}} Informatique \\	
							\textbf{\underline{Parcours:}} Informatique générale\\
							\textbf{\textit{Intitulé}}				\end{center}
				\begin{center}\large\bfseries
			}
			\preauthor{\begin{flushleft}\fontsize{12} \lineskip Présenté le 01 février 2023 }
			\postauthor{\end{flushleft} \textbf{Membres du Jury:} 
			 \begin{itemize}
			    \item \textbf{Président:} Monsieur RALAIVAO Jean Christian, Assistant d'Enseignement Supérieur et de Recherche;
			    \item \textbf{Examinateur:} Monsieur RALAIVAO Jean Christian, Assistant d'Enseignement Supérieur et de Recherche;
			    \item \textbf{Rapporteurs:} \begin{itemize}
       									 \item Monsieur RALAIVAO Jean Christian, Assistant d'Enseignement Supérieur et de Recherche;
       									 \item Monsieur RALAIVAO Jean Christian, Assistant d'Enseignement Supérieur et de Recherche.
    								\end{itemize}
			\end{itemize} }
			\predate{\begin{flushright} Année Universitaire: }
			\postdate{\end{flushright}}
			\title{
				\color{blue}
				\setlength{\fboxsep}{10pt} 
				\fbox{
					\begin{minipage}{0.95\textwidth}
						\begin{center}
							CONCEPTION ET REALISATION  D'UNE APPLICATION WEB  DE RESERVATION DE VOYAGE
						\end{center}
					  \end{minipage}
				}
			}
			\author{\textbf{Par:} Monsieur ANDRIAMIORA Ainamalala Lucky}
			\date{2024-2025}
			\maketitle
			
			\newpage
			\thispagestyle{empty}
			\mbox{}

			\newpage
			\pagenumbering{roman}
			\renewcommand{\thepage}{\Roman{page}} % Ensure uppercase Roman numerals
			\setcounter{page}{1}
			\chapter*{CURRICULUM VITAE}
			\addcontentsline{toc}{chapter}{CURRICULUM VITAE}	
			\begin{minipage}{0.6\textwidth}
				\textbf{Nom:} ANDRIAMIORA\\
				\textbf{Prenom:} Ainamalala Lucky\\
				\textbf{Numéro:} +261 34 33 513 61\\
				\textbf{Addresse:} IIG20 E Ambatomaro, Antananarivo\\
				\textbf{E-mail:} luckyainamalalalucky@gmail.com\\
				\textbf{Date et lieu de naissance:} 30 décembre 1999 à Antsirabe
			\end{minipage}
			\hfill
			\begin{minipage}{0.3\textwidth}
				\includegraphics[width=3.5cm, height=3.5cm]{mypic.jpg}	
			\end{minipage}
			\section*{FORMATIONS ET DIPLOME}
			\begin{minipage}{\textwidth}
				\textbf{2024-2025:} Deuxième année de formation en Master professionnelle à l’École Nationale d'Informatique, Université de Fianarantsoa, parcours : Informatique générale.\\[0.5cm]
				\textbf{2023-2024:} Première année de formation en Master professionnelle à l’École Nationale d'Informatique, Université de Fianarantsoa, parcours : Informatique générale.\\[0.5cm]
				\textbf{2022-2023:} Obtention du diplôme de Licence professionnelle mention Bien à l’École Nationale d'Informatique, Université de Fianarantsoa, parcours : Informatique générale.\\[0.5cm]
				\textbf{2021-2022:} Deuxième année de formation en Licence professionnelle à l’École Nationale d'Informatique, Université de Fianarantsoa, parcours : Informatique générale.\\[0.5cm]
				\textbf{2020-2021:} Première année de formation en Licence professionnelle à l’École Nationale d'Informatique, Université de Fianarantsoa, parcours : Informatique générale.\\[0.5cm]
				\textbf{2019-2020:} Obtention du diplôme de Baccalauréat série D mention assez-bien au Lycée André Résampa Antsirabe.
			\end{minipage}
			\section*{STAGES ET EXPERIENCES PROFESSIONNELLES}
				\begin{minipage}{\textwidth}
       					 \textbf{11 juin 2024 au 4 novembre 2024:} Stage auprès de Nexitia technologies.
						\begin{itemize}
							\item Thème du stage: Application web Handeha voyage;
							\item Langages et outils: Java, spring boot, JavaScript, next js, Katappult, H2, UML.						
						\end{itemize}
       					 \textbf{Octobre 2023:} Projet à l’École Nationale d'Informatique.
						\begin{itemize}
							\item Thème du stage: Module de securité pour application spring boot et angular;
							\item Langages et outils: Java, spring boot, Angular, TypeScript, PostgreSQL, UML.						
						\end{itemize}       					 
					\textbf{19 octobre 2022 au 16 janvier 2023:} Stage auprès de la Paositra malagasy.
						\begin{itemize}
							\item Thème du stage: Application web pour la gestion des change;
							\item Langages et outils: Java, spring boot, Angular, TypeScript, PostgreSQL, UML.								
						\end{itemize}
					\textbf{Septembre 2022:} Projet à l’École Nationale d'Informatique.
						\begin{itemize}
							\item Thème du stage: Création d'une application mobile de gestion de cotisation de groupe;
							\item Langages et outils: Java, Android Studio.								
						\end{itemize}
					\textbf{8 mars 2021 au 7 juillet 2021:} Stage auprès du Ministère de l’Économie et de la Finance.
						\begin{itemize}
							\item Thème du stage: Application web pour la gestion des dossier du personnels de la Direction du Système d'Information;
							\item Langages et outils: Java, spring boot, Angular, TypeScript, PostgreSQL, UML.								
						\end{itemize} 
					\textbf{Avril 2021:} Projet à l’École Nationale d'Informatique.
						\begin{itemize}
							\item Thème du stage: Réalisation d’une application desktop de Gestion des heures complémentaires;
							\item Langages et outils: Php, MySQL, Ajax.								
						\end{itemize} 
					\textbf{Août 2020:} Projet à l’École Nationale d'Informatique.
						\begin{itemize}
							\item Thème du stage: Création d'une application desktop de gestion de stock;
							\item Langages et outils: C++, Qt Creator.								
						\end{itemize} 
   				\end{minipage}		
			\section*{CONNAISSANCES EN INFORMATIQUE}
			\begin{minipage}{\textwidth}
				\begin{itemize}
					\item \textbf{Systèmes d'exploitation:} Microsoft Windows, Linux;
					\item \textbf{Langages de Programmation:} Java, JavaScript, TypeScript, Python , C/C++;
					\item \textbf{Développement Web:} HTML5, CSS3, Tailwind CSS, React.js, Next.js, Angular, Node.js, Express.js;
					\item \textbf{Développement Backend:} Spring Framework, Node.js, Express.js;
					\item \textbf{Compétences en bases de données:} SQL, PostgreSQL, MySQL;
					\item \textbf{Outils de Gestion de Versions:} Git, GitHub, GitLab; 	
					\item \textbf{Outils DevOps:} Docker, Kubernetes, Jenkins, Ansible;
					\item \textbf{Outils de virtualisation:} Docker, VirtualBox, VMware, Vagrant; 	
					\item \textbf{Outils de tests et Assurance Qualité:} Jest, JUnit;
					\item \textbf{Outils et Méthodologies de Conception et Gestion de Projets Logiciels:} UML, Méthodologie Agile, 2TUP, MERISE, GRASP;
					\item \textbf{Outils de Documentation et de Préparation de Contenu:} LaTeX/TeX.	
				\end{itemize}

   			\end{minipage}		
			\section*{CONNAISSANCES LINGUISTIQUE}
			\begin{minipage}{\textwidth}
				\begin{center}
					\begin{tabularx}{\textwidth}{|c|X|X|X|X|}
						\hline
						\textbf{LANGUE} & \textbf{COMPREHENSION} & \textbf{LECTURE} & \textbf{PARLE} & \textbf{ECRIT}\\
						\hline
						Anglais & Bien & Bien & Assez-bien & Bien \\
						\hline
						Français & Bien & Bien & Bien & Bien\\
						\hline
					\end{tabularx}
				\end{center}
			\end{minipage}
			\section*{LOISIR ET CENTRES D'INTERET}
			\begin{minipage}{\textwidth}
				\begin{itemize}
					\item Musique;
					\item Podcast;
					\item Documentaires;
					\item Anime;
					\item Jeux video;
					\item	Lecture.
				\end{itemize}
			\end{minipage}	
			\chapter*{REMERCIEMENTS}
			\addcontentsline{toc}{chapter}{REMERCIEMENTS}	
			\begin{minipage}{\textwidth}
				Avant toute chose, je tiens à remercier Dieu tout puissant et miséricordieux, qui m'a donné la force et la patience d'accomplir ce modeste travail.\\[0.5cm]
				Mes remerciements s’étendent également à :
				\begin{itemize}
					\item Monsieur HAJALALAINA Aimé Richard, Docteur HDR, Président de l'Université de Fianarantsoa, pour tout ce qu'il entreprend à l'Université;
					\item Monsieur MAHATODY Thomas, Docteur HDR, Directeur de l’École Nationale d'Informatique, qui nous à donné l'opportunité d'aller en stage pour ainsi permettre d’accroître nos compétences;
					\item Monsieur RANARISON Richard, Directeur Général de la Paositra Malagasy, pour son accueil et la confiance qu'il à accordée depuis mon arrivée dans l’établissement;
					\item Monsieur RATIARSON Venot, Maître de Conférences, mon encadreur pédagogique, pour ses conseils et échanges tout au long du stage;
					\item Madame RANDRIAMIHARISOA Rollande, mon encadreur professionnelle, qui m'a toujours guidée lors de la réalisation du projet;
					\item Monsieur RALAIVAO Jean Christian, Assistant d'Enseignement Supérieur et de Recherche,d’avoir accepté de présider la soutenance;
				 	\item Monsieur DIMBISOA William Germain, Docteur en Informatique, d'avoir accepté d'examiner mon présent travail;
					\item Toutes les personnels de la Paositra Malagasy, pour leurs accueils;
					\item Toutes les enseignants et les personnels de l’École Nationale d'Informatique qui se sont acharné à nous former durant l'année universitaire;
					\item Ma famille pour son soutien, que se soit moral, matériel ou financier.
				\end{itemize}
			\end{minipage}
			\newpage
			\tableofcontents
			\newpage
			\chapter*{NOMENCLATURE}
			\addcontentsline{toc}{chapter}{NOMENCLATURE}	
			\newpage
			\listoftables
			\newpage
			\listoffigures
			\newpage
			\newpage
			\pagenumbering{arabic}
			\setcounter{page}{1}
			\chapter*{INTRODUCTION GÉNÉRALE}
			\addcontentsline{toc}{chapter}{INTRODUCTION GÉNÉRALE}
			\begin{minipage}{\textwidth}
							
			\end{minipage}
			\newpage
\end{document}